\documentclass[12pt]{article}
\setlength{\textheight}{9.80in}
\setlength{\textwidth}{6.40in}
\setlength{\oddsidemargin}{0.0mm}
\setlength{\evensidemargin}{1.0mm}
\setlength{\topmargin}{-0.6in}
\setlength{\parindent}{0.2in}
\setlength{\parskip}{1.5ex}
\newtheorem{defn}{Definition}
\renewcommand{\baselinestretch}{1.2}

\begin{document}

\thispagestyle{empty}

\title{Topology on a Sphere Using a Truncated Icosahedron}

\maketitle

\tableofcontents

\section{Introduction}

We shall start off with a "template" pentagon, then introduce the nodes and
faces of a dodecahedron. (There is an article in "MathSchool" here discussing
the construction of the the dodecahedron discussed here. From this dodecahedron,
we construct an icosahedron, and form the "bucky-ball" or truncated icosahedron
from that. This in turn we decompose into triangles.

We go into some detail as to node numbers, face numbers, edge numbers, and the
 connections between the faces of the polyhedra. The main purpose of this article
is to accompany a computer code, explaining all the details on nodes, faces, edges
and connections between faces.


\section{A Template Pentagon}

We start off with a template pentagon. For the actual positions of the vertices
 in terms of Golden Ratios, see "MathSchool" or just use very basic trig.
\vspace*{10cm}
\begin{figure}[htb]
\special{psfile=Template.eps vscale=50 hscale=50 voffset=10 hoffset=100}
\caption{ We envisage a template pentagon centred at the origin in the 
($x,y$) plane. Any pentagon shall use this convention for node and edge numbering.
Note the normal (edge 1 vector  cross edge 2 vector) points in the positive $z$
direction. So if this is used as a "base" for a dodecahedron, the normal is an {\it inward
normal.}}
\end{figure}

We shall add a pentagon to each edge of a "base" pentagon in the $(x,y)$ plane and
"fold them up" as discussed in "MathSchool". This shall give us the "bottom half"
of a dodecahedron. We shall insist all out normals be inward pointing. (It makes
no real difference as to whether the normals are inward or outward, however they
must either be {\it all} inward or {\it all} outward. When it comes to the upper half
we must have a mirror image template pentagon as shown below.

\vspace*{10cm}
\begin{figure}[htb]
\special{psfile=Template2.eps vscale=50 hscale=50 voffset=10 hoffset=100}
\caption{ A mirror image template for the "upper half" of a dodecahedron.
}
\end{figure}

\section{The Dodecahedron}

So, we start with a base pentagon (face $A$), We "glue" identical pentagons
to the edges, and "fold them up" so that there are no gaps between the outer ring 
pentagons. This gives us a "lower half" for a dodecahedron. We have six faces
and fifteen nodes numbered as below.

\vspace*{10cm}
\begin{figure}[htb]
\special{psfile=dec1.eps vscale=50 hscale=50 voffset=10 hoffset=100}
\caption{ 
Node numbering for pentagons $A$, $B$, $C$, $D$, $E$ and $F$. Here we shall use
the first template in Fig.1.
}
\end{figure}

Next we swap to the mirror template, and rotate it through 180$\circ$. We glue
on five extra pentagons round the edges and fold them down. This forms the 
upper half of the dodecahedron.

\vspace*{15cm}
\begin{figure}[htb]
\special{psfile=dec2.eps vscale=50 hscale=50 voffset=30 hoffset=100}
\caption{ 
Node numbering for pentagons $G$, $H$, $I$, $J$, $K$ and $L$. Here we use
the second "mirror template" in Fig.2.
}
\end{figure}

First, we write down the nodes of the first six pentagons labelled $A$ to $F$
which have the same convention as in Fig.1.
\begin{center}
\begin{tabular}{| c | c | c | c | c | c |}
\hline
Pentagon &  Node 1 & Node 2 & Node 3 & Node 4 & Node 5 \\
\hline
A & 1 & 2 & 3 & 4 & 5 \\
B & 6 & 7 & 8 & 2 & 1 \\
C &  8 & 9 & 10 & 3 & 2 \\
D & 10 & 11 & 12 & 4 & 3 \\
E & 12 & 13 & 14 & 5 & 4 \\
F & 13 & 15 & 6 & 1 & 5 \\
\hline
\end{tabular}
\end{center}

The remaining pentagons $G$ to $L$ are labelled using the mirror image of Fig.2.
Then the cross product ${\bf e}_1 \times {\bf e}_2$ is always an {\it inward} normal.
Just reverse for outward normals. Here $\bf e$ stands for an edge vector, and
 the subscript is just an edge number. Here are the nodes of the upper half
of the dodecahedron.

\begin{center}
\begin{tabular}{| c | c | c | c | c | c |}
\hline
Pentagon &  Node 1 & Node 2 & Node 3 & Node 4 & Node 5 \\
\hline
G & 9 & 8 & 7 & 20 & 19 \\
H & 11 & 10 & 9 & 19 & 18 \\
I & 13 & 12 & 11 & 18 & 17 \\
J & 15 & 14 & 13 & 17 & 16 \\
K & 7 &  6 & 15 & 16  & 20 \\
L & 16  & 17 & 18 & 19  & 20 \\
\hline
\end{tabular}
\end{center}

Naturally enough, edges 1 to 5 go from nodes 1 to 2, 2 to 3, 3 to 4, 4 to 5, and 5 to 1
in the template pentagon. We shall tabulate which pentagons are connected to 
edges 1 to 5 for each pentagon. We shall use subscripts. The subscript number
matches the edge number in the neighbouring pentagon. For instance
for pentagon A, the first entry is $B_4$. This means that edge 1 of $A$ is connected
 to pentagon $B$. The subscript 4 means that this edge in $A$ corresponds to edge 4 in $B$.Note that nodes are swapped in the neighbouring edge. Edge 1 of $A$ connects to edge 4 of
$B$, but edge 1 in $A$ is $(node 1, node 2)$ while edge 4 in $B$ is $(node 2 node 1)$.


\begin{center}
\begin{tabular}{| c | c | c | c | c | c |}
\hline
Pentagon &  Edge 1 & Edge 2 & Edge 3 &  Edge 4 & Edge 5 \\
\hline
A  & $B_{4}$ & $C_{4}$ & $D_{4}$ & $E_{4}$ & $F_{4}$ \\
B  & $K_{1}$ & $G_{2}$ & $C_{5}$ & $A_{1}$ & $F_{3}$ \\
C  & $G_{1}$ & $H_{2}$ & $D_{5}$ & $A_{2}$ & $B_{3}$ \\
D  & $H_{1}$ & $I_{2}$ & $E_{5}$ & $A_{3}$ & $C_{3}$ \\
E  & $I_{1}$ & $J_{2}$ & $F_{5}$ & $A_{4}$ & $D_{3}$ \\
F  & $J_{1}$ & $K_{2}$ & $B_{5}$ & $A_{5}$ & $E_{3}$ \\
G  & $C_{1}$ & $B_{2}$ & $K_{5}$ & $L_{4}$ & $H_{3}$ \\
H  & $D_{1}$ & $C_{2}$ & $G_{5}$ & $L_{3}$ & $I_{3}$ \\
I  & $E_{1}$ & $D_{2}$ & $H_{5}$ & $L_{2}$ & $J_{3}$ \\
J  & $F_{1}$ & $E_{2}$ & $I_{5}$ & $L_{1}$ & $K_{3}$ \\
K  & $B_{1}$ & $F_{2}$ & $J_{5}$ & $L_{5}$ & $G_{3}$ \\
L  & $K_{4}$ & $J_{4}$ & $I_{4}$ & $H_{4}$ & $G_{4}$ \\
\hline
\end{tabular}
\end{center}

We could go on to split the pentagons into triangles, but instead we shall move on to
the icosahedron.

\section{The Icosahedron}

The Icosahedron is the {\it Platonic Dual} of the dodecahedron. Instead of 20
vertices and 12 faces there are 12 vertices and 20 faces, each of which is an
equilateral triangle. The vertices are at the centres of each pentagon of the
 dodecahedron. In this section, the letters $A$ to $L$ shall represent the position
vectors of the nodes at the centres of pentagons $A$ to $L$. It should be apparent from
the context whether $A$ is pentagon or a node. Again, a triangle has nodes
($N1, N2, N3)$ going  and edges $ (N1,N2), (N2,N3), (N3,N1)$. the cross product
 $edge 1 \times edge 2$ shall be an {\it inward} pointing normal. Without further ado
we can tabulate the nodes of the 20 triangles $T_1$ through to $T_20$. (See Figs.3 and 4.)

\begin{center}
\begin{tabular}{| c | c | c | c | }
\hline
Triangle &  Node 1& Node 2 & Node 3  \\
\hline
  T1& $A$ & $B$ & $C$  \\
  T2& $A$ & $C$ & $D$  \\
  T3& $A$ & $D$ & $E$  \\
  T4& $A$ & $E$ & $F$  \\
  T5& $A$ & $F$ & $B$  \\
  T6& $B$ & $G$ & $C$  \\
  T7& $C$ & $H$ & $D$  \\
  T8& $D$ & $I$ & $E$  \\
  T9& $E$ & $J$ & $F$  \\
  T10& $F$ & $K$ & $B$  \\
  T11& $B$ & $G$ & $K$  \\
  T12& $C$ & $H$ & $G$  \\
  T13& $D$ & $I$ & $H$  \\
  T14& $E$ & $J$ & $I$  \\
  T15& $F$ & $K$ & $J$  \\
  T16& $L$ & $J$ & $I$  \\
  T17& $L$ & $K$ & $J$  \\
  T18& $L$ & $G$ & $K$  \\
  T19& $L$ & $H$ & $G$  \\
  T20& $L$ & $I$ & $H$  \\
\hline
\end{tabular}
\end{center}

We shall also write down how the triangles are connected. In the following, an
entry such as $T_8(3)$ means edge 3 of triangle 8. Any minus sign indicates that the 
nodes of the are swapped. So, in the table below, edge 1 of triangle one connects
to edge 2 of the triangle 5, but the nodes are the opposite way round. Here is
the full list.
\begin{center}
\begin{tabular}{| c | c | c | c | }
\hline
Triangle &  Edge 1& Edge 2 & Edge 43  \\
\hline
  T1& $-T5(3)$ & $-T6(3)$ & $-T2(1)$  \\
  T2& $-T1(3)$ & $-T7(3)$ & $-T3(1)$  \\
  T3& $-T2(3)$ & $-T8(3)$ & $-T4(1)$  \\
  T4& $-T3(3)$ & $-T9(3)$ & $-T5(1)$  \\
  T5& $-T4(3)$ & $-T10(3)$ & $-T1(1)$  \\
  T6& $T11(1)$ & $T12(3)$ & $-T1(2)$  \\
  T7& $T12(1)$ & $T13(3)$ & $-T2(2)$  \\
  T8& $T13(1)$ & $T14(3)$ & $-T3(2)$  \\
  T9& $T14(1)$ & $T15(3)$ & $-T4(2)$  \\
  T10& $T15(1)$ & $T11(3)$ & $-T5(2)$  \\
  T11& $T6(1)$ & $T18(2)$ & $T10(2)$  \\
  T12& $T7(1)$ & $T19(2)$ & $T6(2)$  \\
  T13& $T8(1)$ & $T20(2)$ & $T7(2)$  \\
  T14& $T9(1)$ & $T16(2)$ & $T8(2)$  \\
  T15& $T10(1)$ & $T17(2)$ & $T9(2)$  \\
  T16& $-T17(3)$ & $T14(2)$ & $-T20(1)$  \\
  T17& $-T18(3)$ & $T15(2)$ & $-T16(1)$  \\
  T18& $-T19(3)$ & $T11(2)$ & $-T17(1)$  \\
  T19& $-T20(3)$ & $T12(2)$ & $-T18(1)$  \\
  T20& $-T16(3)$ & $T13(2)$ & $-T19(1)$  \\
\hline
\end{tabular}
\end{center}

\section{Buckminster!}

Now, we shall want to construct a truncated icosahedron. There are five "spokes"
at each node. If we go to any node, and then mark off one third of the length
of the spokes from that node, we have the nodes of small pentagon. For instance,
 node $A$ in the icosahedron is at the centre of pentagon $A$ in the original
dodecahedron. The spokes radiating from node $A$ radiate to nodes $B$ through to $F$.
If we mark off these one third lengths, we have the nodes of a pentagon which
looks like the template pentagon rotated through 180$^circ$ but smaller.

We shall abuse notation a little here for the sake of brevity. By pentagon
$A^\prime$ = nodes $B$, $C$, $D$, $E$ $F$ we mean that node 1 is
 $\stackrel{\longrightarrow}{A}+1/3 \stackrel{\longrightarrow}{AB}$. That is, the
prime denotes the centre, and the letter denotes the node toward which the
spoke is pointing to. The divisions by 3 shall still be there, but they shall be
implicit. The 60 nodes of these twelve smaller pentagons are {\it all} of the nodes
of the truncated icosahedron. This happens to be the structure of the $C60$ molecule,
which is also known as Buckminsterfullerene after the architect Buckminster who
utilised the such structures in geodesic domes. These are 20 hexagons sharing these nodes. Each pentagon
shares edges with five hexagons, and each hexagon shares nodes with three pentagons
and three other hexagons.
The twelve pentagons, using the notation just described are given next.

\begin{center}
\begin{tabular}{| c | c | c | c | c | c |}
\hline
Pentagon &  Node 1& Node 2 & Node 3  & Node 4 & Node 5 \\
\hline
A$^\prime$ & B  & C & D & E & F \\
B$^\prime$ & A  & F & K & G & C \\
C$^\prime$ & A  & B & G & H & D \\
D$^\prime$ & A  & C & H & I & E \\
E$^\prime$ & A  & D & I & J & F \\
F$^\prime$ & A  & E & J & K & B \\
G$^\prime$ & L  & H & C & B & K \\
H$^\prime$ & L  & I & D & C & G \\
I$^\prime$ & L  & J & E & D & H \\
J$^\prime$ & L  & K & F & E & I \\
K$^\prime$ & L  & G & B & F & J \\
L$^\prime$ & K  & J & I & H & G \\
\hline
\end{tabular}
\end{center}

We use the node listings above to define edges 1 to 5 of each pentagon in the truncated icosahedron. That is
edge 1 consists of nodes 1 and 2, edge 2 consists of nodes 2 and 3, up until edge 5
which consists of nodes 5 and 1. We go back to the lower half of the dodecahedron in
Fig.3. Fig.4 below shows a rough sketch of pentagons $A$, $B$ and $C$. Within each
of these is a smaller pentagon which is part of the truncated icosahedron, as marked
by the prime. Triangle $T_1$ is shown. The three primed pentagons "shave off" the edges
of $T_1$ to form the hexagon shown in the thicker line. Each triangle in the icosahedron
is now a hexagon in the truncated icosahedron. We will want to list which edge
of which pentagon corresponds to which edge of which hexagon.

\vspace*{10cm}
\begin{figure}[htb]
\special{psfile=Hex1.eps vscale=50 hscale=50 voffset=10 hoffset=100}
\caption{ 
Triangle $T_1$ is formed by the centres of pentagons $A$, $B$, and $C$. 
These spokes between the pentagon centres define the pentagons $A^\prime$,
 $B^\prime$, and $C^\prime$. One edge of each of these pentagons define the hexagon in
$T_1$.
}
\end{figure}

It is natural to start off by defining the hexagons in terms of the pentagons listed above.We proceed by defining a hexagon in terms of three pentagon edges. (A different pentagon for each edges. We shall denote these as hexagon edges 1, 3, and 5.  The edges
numbered 2, 4, and 6 follow automatically. The nodes and edges of a hexagon in some
triangle $X, Y, Z$ are depicted in Fig.6.
\vspace*{10cm}
\begin{figure}[htb]
\special{psfile=Hex2.eps vscale=50 hscale=50 voffset=10 hoffset=100}
\caption{ 
The hexagon nodes and edges in a triangle with nodes $(X,Y,Z)$ are defined.
}
\end{figure}

In this way, we define the twenty hexagons below. To make the notation clear,
the first line means that edge 1 of hexagon 1 is an edge in pentagon $A^\prime$
In $A^prime$ it is consists of the node pair $(C,B)$ which is edge 1, but in
reverse order, so it is given a minus sign. We shall split the tables up into four 
groups of five


\begin{center}
\begin{tabular}{| c | c | c | c | c | c |}
\hline
Hexagon  & edge & Pentagon & Node 1 & Node 2 & edge \\
\hline
H1 &  e1  & $A^\prime$ &  C  & B &  -e1 \\
   &  e3  & $B^\prime$ &  A  & C &  -e5 \\
   &  e5  & $C^\prime$ &  A  & B &  +e1 \\
\hline
H2 &  e1  & $A^\prime$ &  D  & C &  -e2 \\
   &  e3  & $C^\prime$ &  A  & D &  -e5 \\
   &  e5  & $D^\prime$ &  A  & C &  +e1 \\
\hline
H3 &  e1  & $A^\prime$ &  E  & D &  -e3 \\
   &  e3  & $D^\prime$ &  A  & E &  -e5 \\
   &  e5  & $E^\prime$ &  A  & D &  +e1 \\
\hline
H4 &  e1  & $A^\prime$ &  F  & E &  -e4 \\
   &  e3  & $E^\prime$ &  A  & F &  -e5 \\
   &  e5  & $F^\prime$ &  A  & E &  +e1 \\
\hline
H5 &  e1  & $A^\prime$ &  B  & F &  -e5 \\
   &  e3  & $F^\prime$ &  A  & B &  -e5 \\
   &  e5  & $B^\prime$ &  A  & F &  +e1 \\
\hline
\end{tabular}
\end{center}


\begin{center}
\begin{tabular}{| c | c | c | c | c | c |}
\hline
Hexagon  & edge & Pentagon & Node 1 & Node 2 & edge \\
\hline
H6 &  e1  & $B^\prime$ &  C  & G &  -e4 \\
   &  e3  & $G^\prime$ &  B  & C &  -e3 \\
   &  e5  & $C^\prime$ &  B  & G &  +e2 \\
\hline
H7 &  e1  & $C^\prime$ &  D  & H &  -e4 \\
   &  e3  & $H^\prime$ &  C  & D &  -e3 \\
   &  e5  & $D^\prime$ &  C  & H &  +e2 \\
\hline
H8 &  e1  & $D^\prime$ &  E  & I &  -e4 \\
   &  e3  & $I^\prime$ &  D  & E &  -e3 \\
   &  e5  & $E^\prime$ &  D  & I &  +e2 \\
\hline
H9 &  e1  & $E^\prime$ &  F  & J &  -e4 \\
   &  e3  & $J^\prime$ &  E  & F &  -e3 \\
   &  e5  & $F^\prime$ &  E  & J &  +e2 \\
\hline
H10 &  e1  & $F^\prime$ &  B  & K &  -e4 \\
   &  e3  & $K^\prime$ &  F  & B &  -e3 \\
   &  e5  & $B^\prime$ &  F  & K &  +e2 \\
\hline
\end{tabular}
\end{center}

\begin{center}
\begin{tabular}{| c | c | c | c | c | c |}
\hline
Hexagon  & edge & Pentagon & Node 1 & Node 2 & edge \\
\hline
H11 &  e1  & $B^\prime$ &  G  & B &  -e2 \\
   &  e3  & $G^\prime$ &  K  & B &  -e4 \\
   &  e5  & $K^\prime$ &  G  & K &  -e3 \\
\hline
H12 &  e1  & $C^\prime$ &  C  & H &  -e2 \\
   &  e3  & $H^\prime$ &  G  & C &  -e4 \\
   &  e5  & $G^\prime$ &  H  & G &  -e3 \\
\hline
H13 &  e1  & $D^\prime$ &  D  & I &  -e2 \\
   &  e3  & $I^\prime$ &  H  & D &  -e4 \\
   &  e5  & $H^\prime$ &  I  & H &  -e3 \\
\hline
H14 &  e1  & $E^\prime$ &  E  & J &  -e2 \\
   &  e3  & $J^\prime$ &  I  & E &  -e4 \\
   &  e5  & $I^\prime$ &  J  & I &  -e2 \\
\hline
H15 &  e1  & $F^\prime$ &  F  & K &  -e2 \\
   &  e3  & $K^\prime$ &  J  & F &  -e4 \\
   &  e5  & $J^\prime$ &  K  & J &  -e3 \\
\hline
\end{tabular}
\end{center}

\begin{center}
\begin{tabular}{| c | c | c | c | c | c |}
\hline
Hexagon  & edge & Pentagon & Node 1 & Node 2 & edge \\
\hline
H16 &  e1  & $L^\prime$ &  I  & J &  +e3 \\
   &  e3  & $J^\prime$ &  L  & I &  -e5 \\
   &  e5  & $I^\prime$ &  L  & J &  +e1 \\
\hline
H17 &  e1  & $L^\prime$ &  J  & K &  +e4 \\
   &  e3  & $K^\prime$ &  L  & J &  -e5 \\
   &  e5  & $J^\prime$ &  L  & K &  +e1 \\
\hline
H18 &  e1  & $L^\prime$ &  K  & G &  +e5 \\
   &  e3  & $G^\prime$ &  L  & K &  -e5 \\
   &  e5  & $K^\prime$ &  L  & G &  +e1 \\
\hline
H19 &  e1  & $L^\prime$ &  G  & H &  +e1 \\
   &  e3  & $H^\prime$ &  L  & G &  -e5 \\
   &  e5  & $G^\prime$ &  L  & H &  +e1 \\
\hline
H20 &  e1  & $L^\prime$ &  H  & I &  +e2 \\
   &  e3  & $I^\prime$ &  L  & H &  -e5 \\
   &  e5  & $H^\prime$ &  L  & I &  +e1 \\
\hline
\end{tabular}
\end{center}

We can write down the inverse table, which tells us which hexagon lies
across any edge of any pentagon, and what the corresponding  edge number and edge
orientation is in the hexagon. 


The question as to how the connectivity between the hexagons has already
been done. The connections are the same as for the triangles in the icosahedron
 section, except that triangle edges 1, 2, and 3, are now hexagon edges 2, 4, and 5.
So we have the nodes of the truncated icosahedron, the nodes of each pentagon and
connection, how the pentagons and hexagons fit to each other, and how the hexagons
fit with each other.



\end{document}

